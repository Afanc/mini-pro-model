\documentclass[10pt,a4paper,oneside,twocolumn]{article}

    \usepackage{float}	% for floating figures (putting them anywhere we want
    \usepackage{amsmath}	% for maths
    \usepackage{graphicx}	% jpg
    \usepackage{hyperref}
    \usepackage{textcomp}
    \usepackage{verbatim}	% use of \begin{comment}
    \usepackage{pgfplots}	% use of pgf plots
    \usepackage{multicol}
    \usepackage[top=2cm, bottom=2cm, left=1.5cm, right=1.5cm]{geometry} %margins
    \usepackage{sidecap}	% for side captions
    \restylefloat{table}	% floating figures(Tables)
    \usepackage{caption}
    \captionsetup{justification=centering}
    \usepackage{subcaption}
    \usepackage{array}
    \usepackage{titlesec}
    \titleformat{\section}[block]{\Large\bfseries\filcenter}{}{1em}{}

    \newcommand{\red}[1]{\textcolor{red}{#1}} 
    \newcommand{\cyan}[1]{\textcolor{cyan}{#1}} 
    \newcommand{\blue}[1]{\textcolor{blue}{#1}} 
    \newcommand{\orange}[1]{\textcolor{orange}{#1}} 
    \newcommand{\purple}[1]{\textcolor{purple}{#1}} 
    \newcommand{\green}[1]{\textcolor{green}{#1}} 
    \definecolor{forestgreen}{RGB}{0,102,0}
    \newcommand{\fgreen}[1]{\textcolor{forestgreen}{#1}} 

    \numberwithin{equation}{section} %permits numbering within sections instead of globally
\begin{document}

\title{\huge{\textbf{Mini-Project in Mathematical and Computational Modeling}}\\
	\vspace{0.5cm}
	\Large{\textit{\'Ecole Polytechnique F\'ed\'erale de Lausanne, Switzerland}}}
\author{\large{Florian + Dariush}}
\twocolumn[{
    \centering
    \huge{\textbf{Mini-Project in Mathematical and Computational Modeling}} \\
    \vspace{0.5cm}
    \Large{\textit{\'Ecole Polytechnique F\'ed\'erale de Lausanne, Switzerland}}\\
    \vspace{0.5cm}
    \large{Florian + Dariush}
    \vspace{1cm}
	}]

    	\section{Introduction}
    The periodic changes that come about due to the day-night cycle to things ranging from visibility to temperature create an environment in which it is advantageous to adapt behaviour to this extrinsic period. Perhaps vision becomes so restricted in the night that basic processes like hunting for food or looking for mates become so inefficient that it's a better strategy to conserve energy when it's dark. Perhaps the opposite is the case and hunting during the night becomes an advantage due to specially evolved senses in animals such as, for example, bats. \\
    
    For this reason many different lifeforms, from bacteria to animals, have adapted to the 24 hour period that exists around them by synchronising behaviour and processes in different tissues to the day-night cycle by keeping track of time using internal 'clocks'. \\
    
    In humans this clock works by having a part of the brain, called the suprachiasmatic nuclei (SCN), manage the production of several clock proteins in such a way as to create oscillatory patterns in the concentration of these proteins. These patterns are called circadian when they follow the 24-hour period of the earth's rotation. \\
    
    Negative feedback loops are used in such a way as to create a periodic rise and fall in the production of proteins that can be influenced by external factors such a light intensity and food uptake to regularly adjust the synchronisation to new environments. \par

	\section{The Model}
In this simplified model of the SCN we do not take into account such outside signals and instead focus on behaviours of cells in the SCN all by themselves. The model looks at the production of clock gene mRNA (designated as X), which leads to production of a clock protein(Y). This clock gene protein activates transcription of a third compound (Z), which acts as a transcriptional inhibitor of X, thus creating a negative feedback loop. \\
    
    In this report we examine several conditions that need to be met for this simple model to exhibit circadian oscillations in the concentrations of the compounds involved. This model is then expanded to investigate synchronisation between cells by taking into account a third protein (V) that is exported and allows the cells in the SCN to talk to each other. First taking into account only 2 cells, this model is expanded to hundreds of cells later on, not reaching the human population of cells in the SCN of 10'000, but at least providing a smaller model that could be scaled up to that number if more processing power was available. \par
    
    \section{Analysis of the model A}
    The model under investigation in this report will start out very simple, with a large number of shortcomings and assumptions, which we will attempt to improve upon as analysis progresses. At first we consider a single cell, in a very simple attempt to find conditions under which the concentrations of X,Y and Z will display circadian oscillation. In this first part we will especially focus on the central parameter of the base translation rate of X, the clock gene mRNA. We will vary its value and observe under which conditions circadian behaviour appears. This model obviously looks at one cell in isolation and fails to capture the main usefulness of the system, its ability to integrate information from outside. \par
    
    \section{Analysis of the model B}
    In section B we will include multiple cells and the production of the protein V, which allows cells to communicate and synchronise. In this section we will assume that V diffuses so quickly that we can assume its concentration to be the same around all the cells we look at, which is obviously a significant simplification. We would also assume all kinds of delays in its mode of action, depending on how its concentration is sensed by the cells. Of special interest in this section are intrinsic periods of cells. It is sensibly assumed that not all cells will have perfectly identical production rates, and that ultimately the sum of all these differences will be apparent in the period of their oscillations. Next to that we will investigate the coupling factor K, which allows us to manipulate how strongly the cells respond to the average extracellular concentration of V, denoted as F. The strength of the coupling factor, as well as the difference in intrinsic periods of the cells have an impact on whether the cells will synchronise or not, and this section attempts to shed light on these respective conditions. 
    
    
    

    \begin{figure*}[!htb] 		%oh my god this is so ugly pls don't look at me
	\captionsetup{labelformat=empty}
	\caption{\Huge{\textbf{Part A - One-Cell Model}}}
    \end{figure*}
    \begin{figure*}[!htb]
	\begin{subfigure}[b]{0.5\textwidth}
	    \includegraphics[width=\textwidth]{sketch.png}
	    \caption{
		One-Cell Model\\
	    The gene mRNA $X$ codes for protein $Y$ which, in turn, activates transcriptional inhibitor $Z$. The resulting model behaves as a three-variable oscillator.
	    }
	\end{subfigure}
	~
	\begin{subfigure}[b]{0.5\textwidth}
	    \begin{equation*}\frac{\delta X}{\delta t} = v_1 \frac{K_1^n}{K_1^n + Z^n} - v_2 \frac{X}{K_2 + X} \end{equation*}
	    \begin{equation*}\frac{\delta Y}{\delta t} = k_3 X - v_4 \frac{Y}{K_4 + Y}\end{equation*}
	    \begin{equation*}\frac{\delta Z}{\delta t} = k_5 Y - v_6 \frac{Z}{K_6 + Z}\end{equation*}

	    \captionsetup{labelformat=empty}
	    \caption{\\
	    \begin{tabular}{@{}>{$}l<{$}l @{\hskip 0.2cm} | @{\hskip 0.2cm} @{}>{$}l<{$}l@{}}
		v_1 & translation rate of $X$ & K_1 & Michaelis constant of $X$ \\
		v_2 & degradation rate of $X$ & K_4 & Michaelis constant of $Y$ \\
		v_4 & degradation rate of $Y$ & K_6 & Michaelis constant of $Z$\\
		v_6 & degradation rate of $Z$ &&\\
		k_3 & transcription rate of $Y$ && \\
		k_5 & transcription rate of $Z$ &&\\
	    \end{tabular}
	    }
	\end{subfigure}
    \end{figure*}

    \begin{figure*}[!h]
	\begin{subfigure}[b]{0.5\textwidth}
	    \includegraphics[width=\textwidth]{"../Miniprojet 2.0/Part A/A11.png}
	    \caption{Trajectories\\
	    The limit cycle is reached as the variations of $X(t)$, $Y(t)$ and $Z(t)$ become fixed : The trajectories converge, non-linearly (the distance between similar trajectories aren't regular) towards an elliptic limit cycle. The limit cycle is reached quickly due to favorable choice of initial conditions close to final concentrations. }
	\end{subfigure}
	~
	\begin{subfigure}[b]{0.5\textwidth}
	    \includegraphics[width=\textwidth]{"../Miniprojet 2.0/Part A/A12.png}
	    \caption{Frequency spectrum \\
	    The amplitude of the three variations stabilize after a few hundred hours. The signal are not in phase but have the same, regular, frequencies.}
	\end{subfigure}
	\caption{\\Trajectories of $X(t)$, $Y(t)$ and $Z(t)$ with initial conditions : $X_0 = 0.16$, $Y_0 = 0.33 $, $Z_0 = 1.8$ [nM]\\ We observe on both graphs that $Z(t)$ has the bigger amplitude of variation whereas $X(t)$ and $Y(t)$ have small amplitudes. Additionally, the convergence towards a single loop in (a) indicate that the frequencies of the signals are equal; this is illustrated as well in (b)
	}
    \end{figure*}

    \begin{figure*}
    \centering
	\begin{subfigure}[b]{0.3\textwidth}
	    \includegraphics[width=\textwidth]{"../Miniprojet 2.0/Part A/A_3_graphs/A-A0.png}
	    \caption{$v_1$ = 0 nM/h}
	    \end{subfigure}
	    ~ 
	    \begin{subfigure}[b]{0.3\textwidth}
	    \includegraphics[width=\textwidth]{"../Miniprojet 2.0/Part A/A_3_graphs/A-A1.png}
	    \caption{$v_1$ = 0.1 nM/h}
	    \end{subfigure}
	    ~ 
	\begin{subfigure}[b]{0.3\textwidth}
	    \includegraphics[width=\textwidth]{"../Miniprojet 2.0/Part A/A_3_graphs/A-A2.png}
	    \caption{$v_1$ = 0.2 nM/h}
	\end{subfigure}
	 
	\begin{subfigure}[b]{0.3\textwidth}
	    \includegraphics[width=\textwidth]{"../Miniprojet 2.0/Part A/A_3_graphs/A-A3.png}
	    \caption{$v_1$ = 0.3 nM/h}
	\end{subfigure}
	~ 
	\begin{subfigure}[b]{0.3\textwidth}
	    \includegraphics[width=\textwidth]{"../Miniprojet 2.0/Part A/A_3_graphs/A-A4.png}
	    \caption{$v_1$ = 0.4 nM/h}
	\end{subfigure}
	~
	\begin{subfigure}[b]{0.3\textwidth}
	    \includegraphics[width=\textwidth]{"../Miniprojet 2.0/Part A/A_3_graphs/A-A6.png}
	    \caption{$v_1$ = 0.6 nM/h}
	\end{subfigure}
	~ 
	\begin{subfigure}[b]{0.3\textwidth}
	    \includegraphics[width=\textwidth]{"../Miniprojet 2.0/Part A/A_3_graphs/A-A8.png}
	    \caption{$v_1$ = 0.8 nM/h}
	\end{subfigure}
	~
	\begin{subfigure}[b]{0.3\textwidth}
	    \includegraphics[width=\textwidth]{"../Miniprojet 2.0/Part A/A_3_graphs/A-A10.png}
	    \caption{$v_1$ = 1 nM/h}
	\end{subfigure}

	\caption{\green{$X(t)$}, \red{$Y(t)$} and \blue{$Z(t)$} with initial conditions $X_0 = 0.16$, $Y_0 = 0.33 $, $Z_0 = 1.8$ [nM]\\
	The first signal to fade is $Y(t)$ and its oscillatory stability predicts stability of the system. We also observe that the signals converge towards null, a fixed point or the limit cycle in a non-linear fashion. It is hard to precisely know the value for $v_1$ for which the system begins to display circadian behaviour, but it must be somewhere around $v_1$ = 0.5nM/h. A more accurate estimate will be made in the bifurcation diagram that follows. Interesting here is that the concentrations will reach a stable fixed point if $v_1$ is below a certain threshold. }
    \end{figure*}

    \begin{figure*}
	\centering
	    \begin{subfigure}[b]{0.35\textwidth}
		\includegraphics[width=\textwidth]{"../Miniprojet 2.0/Part A/Bifurcation.png}
		\caption{at $t_{max}=1000$ h}
	    \end{subfigure}
	     ~ 
	    \begin{subfigure}[b]{0.35\textwidth}
		\includegraphics[width=\textwidth]{"../Miniprojet 2.0/Part A/Bifurcation10000.png}
		\caption{at $t_{max}=10~000$ h}
	    \end{subfigure}
	    \caption{\small{Bifurcation Diagram : \red{$X_{min}$} and \green{$X_{max}$} plotted at time intervals $[9/10; 1]$ of $t_{max}$, meaning it plots the maximum and minimum values in the last tenth of the simulation where we are sure that a either a limit cycle or a stable fixed point has already been reached. A limit cycle might be reached when $X_{min} \neq X_{max}$. However, the system needs to be run for enough time for the cycle to be reached, as the (a) suggests. In figure (b) the simulation was run for ten times longer, showing that for values of $v_1$ between 0.2 and 0.4 nM/h, there is very slow oscillatory convergence to a fixed point, which is the same result that is suggested by figure 4 above. The threshold for $v_1$ for the system to reach circadian oscillation seems to be around 0.4 nM/h}}
    \end{figure*}

    \begin{figure*}
    \centering
	\begin{subfigure}[b]{0.32\textwidth}
	    \includegraphics[width=\textwidth]{"../Miniprojet 2.0/Part A/A_3_graphs/A-AA0.png}
	    \caption{$v_1$ = 0 nM/h}
	\end{subfigure}
	~ 
	\begin{subfigure}[b]{0.32\textwidth}
	    \includegraphics[width=\textwidth]{"../Miniprojet 2.0/Part A/A_3_graphs/A-AA1.png}
	    \caption{$v_1$ = 0.1 nM/h}
	\end{subfigure}
	~ 
	\begin{subfigure}[b]{0.32\textwidth}
	    \includegraphics[width=\textwidth]{"../Miniprojet 2.0/Part A/A_3_graphs/A-AA2.png}
	    \caption{$v_1$ = 0.2 nM/h}
	\end{subfigure}
	 
	\begin{subfigure}[b]{0.32\textwidth}
	    \includegraphics[width=\textwidth]{"../Miniprojet 2.0/Part A/A_3_graphs/A-AA3.png}
	    \caption{$v_1$ = 0.3 nM/h}
	\end{subfigure}
	~ 
	\begin{subfigure}[b]{0.32\textwidth}
	    \includegraphics[width=\textwidth]{"../Miniprojet 2.0/Part A/A_3_graphs/A-AA4.png}
	    \caption{$v_1$ = 0.4 nM/h}
	\end{subfigure}
	~
	\begin{subfigure}[b]{0.32\textwidth}
	    \includegraphics[width=\textwidth]{"../Miniprojet 2.0/Part A/A_3_graphs/A-AA6.png}
	    \caption{$v_1$ = 0.6 nM/h}
	\end{subfigure}
	~ 
	\begin{subfigure}[b]{0.32\textwidth}
	    \includegraphics[width=\textwidth]{"../Miniprojet 2.0/Part A/A_3_graphs/A-AA7.png}
	    \caption{$v_1$ = 0.8 nM/h}
	\end{subfigure}
	~
	\begin{subfigure}[b]{0.32\textwidth}
	    \includegraphics[width=\textwidth]{"../Miniprojet 2.0/Part A/A_3_graphs/A-AA10.png}
	    \caption{$v_1$ = 1 nM/h}
	\end{subfigure}
	
	\caption{Trajectories when varying $v_1$ with initial conditions $X_0 = 0.16$, $Y_0 = 0.33 $, $Z_0 = 1.8$ [nM]\\
	$v_1$ has to reach a certain value for $X(t)$ to be able to compensate its inhibition by $Z(t)$ and therefore for the system to reach a limit cycle. We observe that this value is around 0.4 nM/h, as the trajectories still converge close to zero in (e); there is an 'eye', even though it is smaller than in (f) and (g). It is possible that the simulation time is not long enough to let the system dissipate completely.
	}
	\label{fig:6}
    \end{figure*}

    \begin{figure*}
	\centering
	\begin{subfigure}[b]{0.4\textwidth}
	    \includegraphics[width=\textwidth]{"../Miniprojet 2.0/Part A/A2.png}
	    \caption{Superimposed trajectories at late timepoints with initial conditions $X_0 = 0.16$, $Y_0 = 0.33 $, $Z_0 = 1.8$ [nM] and $v_1$ = \cyan{0.1}/\red{0.3}/\orange{0.5}/\purple{0.7}/\fgreen{0.9} nM/h. We observe here that $Z(t)$ tends to reach greater concentration stability with increasing $v_1$.}
	\end{subfigure}
	~
	\begin{subfigure}[b]{0.4\textwidth}
	    \includegraphics[width=\textwidth]{"../Miniprojet 2.0/Part A/periodogram.png}
	    \caption{Periodogram of the oscillations above with \\ $v_1$ =0.7nM/h.}
	\end{subfigure}
	\caption{This analysis of the trajectories allows to sample for frequencies that are strongly represented in a set of data points. The simulations above appear to have a strong representation around a frequency that corresponds to a period of 23.4 hours, which is close to the 24 hour circadian rhythm that is observed in nature.}
    \end{figure*}

    \begin{figure*}[!htb] 		%oh my god this is so ugly pls don't look at me
	\captionsetup{labelformat=empty}
	\caption{\Huge{\textbf{Part B - Multiple Cells Model}}}
    \end{figure*}

    \begin{figure*}[!h]
	\begin{subfigure}[b]{0.5\textwidth}
	    \includegraphics[width=\textwidth]{sketch2.png}
	    \caption{
		Multiple Cells Model\\
	    The gene $X$ codes for protein $Y$ which, in turn, activates transcriptional inhibitor $Z$. In addition, gene $X$ activates a positive feedback loop through the mean concentration of extracellular protein $V$
	    }
	\end{subfigure}
	~
	\begin{subfigure}[b]{0.5\textwidth}
	    \begin{equation*}\frac{\delta X}{\delta t} = v_1 \frac{K_1^n}{K_1^n + Z^n} - v_2 \frac{X}{K_2 + X} + v_c\frac{KF}{K_c + KF}\end{equation*}
	    \begin{equation*}\frac{\delta Y}{\delta t} = k_3 X - v_4 \frac{Y}{K_4 + Y}\end{equation*}
	    \begin{equation*}\frac{\delta Z}{\delta t} = k_5 Y - v_6 \frac{Z}{K_6 + Z}\end{equation*}
	    \begin{equation*}\frac{\delta V_i}{\delta t} = k_7 X_i - v_8 \frac{V_i}{K_8 + V_i}\end{equation*}
	    \begin{equation*}\text{where } F = \frac{1}{N}\sum_{i=1}^{N}V_i\end{equation*}

	    \captionsetup{labelformat=empty}
	    \caption{\\
	    \begin{tabular}{@{}>{$}l<{$}l @{\hskip 0.2cm} | @{\hskip 0.2cm} @{}>{$}l<{$}l@{}}
		v_1 & translation rate of $X$ & k_1 & transcription rate of $X$ \\
		v_2 & degradation rate of $X$ &	K_1 & Michaelis constant of $X$ \\
		v_4 & degradation rate of $Y$ & K_4 & Michaelis constant of $Y$ \\
		v_6 & degradation rate of $Z$ & K_6 & Michaelis constant of $Z$ \\
		v_8 & degradation rate of $V$ & K_8 & Michaelis constant of $V$ \\
		k_3 & transcription rate of $Y$ & K_c & Michaelis constant of $X$ by $F$\\
		k_5 & transcription rate of $Z$ & v_c & Activation rate of $X$ by $F$ \\
		k_7 & transcription rate of $V$ & K & Coupling Constant \\
	    \end{tabular}
	    }
	\end{subfigure}
    \end{figure*}

    \begin{figure*}
    \centering
	\begin{subfigure}[b]{0.32\textwidth}
	    \includegraphics[width=\textwidth]{"../Miniprojet 2.0/Part B/B_2_graphs/B21.png}
	    \caption{$\lambda_1$ = 1, $\lambda_2$ = 1 [$h^{-1}$]}
	\end{subfigure}
	~ 
	\begin{subfigure}[b]{0.32\textwidth}
	    \includegraphics[width=\textwidth]{"../Miniprojet 2.0/Part B/B_2_graphs/B22.png}
	    \caption{$\lambda_1$ = 1, $\lambda_2$ = 1.2 [$h^{-1}$]}
	\end{subfigure}
	~ 
	\begin{subfigure}[b]{0.32\textwidth}
	    \includegraphics[width=\textwidth]{"../Miniprojet 2.0/Part B/B_2_graphs/B23.png}
	    \caption{$\lambda_1$ = 1, $\lambda_2$ = 1.4 [$h^{-1}$]}
	\end{subfigure}
	 
	\begin{subfigure}[b]{0.32\textwidth}
	    \includegraphics[width=\textwidth]{"../Miniprojet 2.0/Part B/B_2_graphs/B24.png}
	    \caption{$\lambda_1$ = 1, $\lambda_2$ = 1.6 [$h^{-1}$]}
	\end{subfigure}
	~ 
	\begin{subfigure}[b]{0.32\textwidth}
	    \includegraphics[width=\textwidth]{"../Miniprojet 2.0/Part B/B_2_graphs/B25.png}
	    \caption{$\lambda_1$ = 1, $\lambda_2$ = 1.8 [$h^{-1}$]}
	\end{subfigure}
	~ 
	\begin{subfigure}[b]{0.32\textwidth}
	    \includegraphics[width=\textwidth]{"../Miniprojet 2.0/Part B/B_2_graphs/B26.png}
	    \caption{$\lambda_1$ = 1, $\lambda_2$ = 2 [$h^{-1}$]}
	\end{subfigure}

	\caption{\red{$X_1(t)$} and \blue{$X_2(t)$} trajectories in a two-cells Model with $K=0$ ($\leftrightarrow$ no coupling)\\
	Figure (a) has both signals perfectly aligned. We observe no synchronisation, as expected These figures mainly serve to show whether the signals are in or out of phase, to help interpret figure 11 more easily.}
    \end{figure*}


    \begin{figure*}
    \centering
	\begin{subfigure}[b]{0.32\textwidth}
	    \includegraphics[width=\textwidth]{"../Miniprojet 2.0/Part B/B_2_graphs/B11.png}
	    \caption{$\lambda_1$ = 1, $\lambda_2$ = 1 [$h^{-1}$]}
	\end{subfigure}
	~ 
	\begin{subfigure}[b]{0.32\textwidth}
	    \includegraphics[width=\textwidth]{"../Miniprojet 2.0/Part B/B_2_graphs/B12.png}
	    \caption{$\lambda_1$ = 1, $\lambda_2$ = 1.2 [$h^{-1}$]}
	\end{subfigure}
	~ 
	\begin{subfigure}[b]{0.32\textwidth}
	    \includegraphics[width=\textwidth]{"../Miniprojet 2.0/Part B/B_2_graphs/B13.png}
	    \caption{$\lambda_1$ = 1, $\lambda_2$ = 1.4 [$h^{-1}$]}
	\end{subfigure}
	 
	\begin{subfigure}[b]{0.32\textwidth}
	    \includegraphics[width=\textwidth]{"../Miniprojet 2.0/Part B/B_2_graphs/B14.png}
	    \caption{$\lambda_1$ = 1, $\lambda_2$ = 1.6 [$h^{-1}$]}
	\end{subfigure}
	~ 
	\begin{subfigure}[b]{0.32\textwidth}
	    \includegraphics[width=\textwidth]{"../Miniprojet 2.0/Part B/B_2_graphs/B15.png}
	    \caption{$\lambda_1$ = 1, $\lambda_2$ = 1.8 [$h^{-1}$]}
	\end{subfigure}
	~ 
	\begin{subfigure}[b]{0.32\textwidth}
	    \includegraphics[width=\textwidth]{"../Miniprojet 2.0/Part B/B_2_graphs/B16.png}
	    \caption{$\lambda_1$ = 1, $\lambda_2$ = 2 [$h^{-1}$]}
	\end{subfigure}

	\caption{$X_1$ and $X_2$ trajectories with varying $\lambda_i$ in a two-cells Model with $K=0$\\
	Figure (a), the control, makes perfect sense since the two cells have the same period, hence the exact same signal. With unequal periods, the limit cycles of both cells aren't in phase and form these '8' patterns. The fluctuations at the beginning of trajectories come from the inner adjustment of the cells (see Figure \ref{fig:6}). The sides of the rectangles that appear represent the variation of $X_1$ and $X_2$  between their minimal and maximal values and receiving a rectangle shows us that we reach a limit cycle. (f) gives us a different pattern due to the two periods being a multiple of each other. We expect the phase difference not to vary greatly over time, as in the other figures. }
	\label{fig:11}
    \end{figure*}
    
   \begin{figure*}
    \centering
	\begin{subfigure}[b]{0.32\textwidth}
	    \includegraphics[width=\textwidth]{"../Miniprojet 2.0/Part B/B_2_graphs/B41.png}
	    \caption{$\lambda_1$ = 1, $\lambda_2$ = 1 [$h^{-1}$]}
	\end{subfigure}
	~ 
	\begin{subfigure}[b]{0.32\textwidth}
	    \includegraphics[width=\textwidth]{"../Miniprojet 2.0/Part B/B_2_graphs/B42.png}
	    \caption{$\lambda_1$ = 1, $\lambda_2$ = 1.2 [$h^{-1}$]}
	\end{subfigure}
	~ 
	\begin{subfigure}[b]{0.32\textwidth}
	    \includegraphics[width=\textwidth]{"../Miniprojet 2.0/Part B/B_2_graphs/B43.png}
	    \caption{$\lambda_1$ = 1, $\lambda_2$ = 1.4 [$h^{-1}$]}
	\end{subfigure}
	 
	\begin{subfigure}[b]{0.32\textwidth}
	    \includegraphics[width=\textwidth]{"../Miniprojet 2.0/Part B/B_2_graphs/B44.png}
	    \caption{$\lambda_1$ = 1, $\lambda_2$ = 1.6 [$h^{-1}$]}
	\end{subfigure}
	~ 
	\begin{subfigure}[b]{0.32\textwidth}
	    \includegraphics[width=\textwidth]{"../Miniprojet 2.0/Part B/B_2_graphs/B45.png}
	    \caption{$\lambda_1$ = 1, $\lambda_2$ = 1.8 [$h^{-1}$]}
	\end{subfigure}
	~ 
	\begin{subfigure}[b]{0.32\textwidth}
	    \includegraphics[width=\textwidth]{"../Miniprojet 2.0/Part B/B_2_graphs/B46.png}
	    \caption{$\lambda_1$ = 1, $\lambda_2$ = 2 [$h^{-1}$]}
	\end{subfigure}

	\caption{\orange{$X_1(t)$} and \blue{$Y_2(t)$} trajectories in a two-cells Model with $K=0$ (no synchronisation). The key observation to make here is that the signal of Y(t) is always slightly delayed to X(t).\\
	}
    \end{figure*}

    \begin{figure*}
    \centering
	\begin{subfigure}[b]{0.32\textwidth}
	    \includegraphics[width=\textwidth]{"../Miniprojet 2.0/Part B/B_2_graphs/B31.png}
	    \caption{$\lambda_1$ = 1, $\lambda_2$ = 1 [$h^{-1}$]}
	\end{subfigure}
	~ 
	\begin{subfigure}[b]{0.32\textwidth}
	    \includegraphics[width=\textwidth]{"../Miniprojet 2.0/Part B/B_2_graphs/B32.png}
	    \caption{$\lambda_1$ = 1, $\lambda_2$ = 1.2 [$h^{-1}$]}
	\end{subfigure}
	~ 
	\begin{subfigure}[b]{0.32\textwidth}
	    \includegraphics[width=\textwidth]{"../Miniprojet 2.0/Part B/B_2_graphs/B33.png}
	    \caption{$\lambda_1$ = 1, $\lambda_2$ = 1.4 [$h^{-1}$]}
	\end{subfigure}
	 
	\begin{subfigure}[b]{0.32\textwidth}
	    \includegraphics[width=\textwidth]{"../Miniprojet 2.0/Part B/B_2_graphs/B34.png}
	    \caption{$\lambda_1$ = 1, $\lambda_2$ = 1.6 [$h^{-1}$]}
	\end{subfigure}
	~ 
	\begin{subfigure}[b]{0.32\textwidth}
	    \includegraphics[width=\textwidth]{"../Miniprojet 2.0/Part B/B_2_graphs/B35.png}
	    \caption{$\lambda_1$ = 1, $\lambda_2$ = 1.8 [$h^{-1}$]}
	\end{subfigure}
	~ 
	\begin{subfigure}[b]{0.32\textwidth}
	    \includegraphics[width=\textwidth]{"../Miniprojet 2.0/Part B/B_2_graphs/B36.png}
	    \caption{$\lambda_1$ = 1, $\lambda_2$ = 2 [$h^{-1}$]}
	\end{subfigure}

	\caption{$X_1$ and $Y_2$ trajectories with varying $\lambda_i$ in a two-cells Model with $K=0$\\
	Once again, both cells tend to reach their limit cycles without any kind of interaction. The same observations as in Figure \ref{fig:11} can be made, except for the key difference in figure (a), where the fact that the two cells share the same period but are partially out of phase creates a limit cycle instead of a linear interaction.
	}
    \end{figure*}

    \begin{figure*}
    \centering
	    \begin{subfigure}[b]{0.35\textwidth}
	    \begin{equation*}R = \frac{\langle F^2 \rangle - {\langle F \rangle}^2}{\frac{1}{N}\sum_{i=1}^{N}(\langle V_i^2 \rangle - {\langle V_i \rangle}^2)} \end{equation*}
	    \hfill \\
	    \centering
	    \footnotesize{The Coefficient of Synchronization R is as the name suggests a measure of the synchronisation between $N$ different cells. This ratio can take any value between 0 and 1 as the variance of $F$ cannot exceed the mean value of the individual variances of $V_i$. \\
	    $R=1$ means very high synchronisation whereas $R=0$ means no synchronisation at all.}
	    \captionsetup{labelformat=empty}
	    \caption{\hfill \\ \hfill \\ \hfill \\} 	%ok this is freaking lame
	\end{subfigure}
	~
	\begin{subfigure}[b]{0.4\textwidth}
	    \includegraphics[width=\textwidth]{"../Miniprojet 2.0/Part B/B_N_3_graph_2.png}
	    \caption{\footnotesize{Value of $R$ depending on the Coupling Constant K}}
	\end{subfigure}
	\caption{We introduce the Coefficient of Synchronization R \\
	The plot shows some irregularity, but shows a clear trend. The higher the strength of the signal of V is (denoted by the factor K), the more the cells begin to oscillate in synchronisation.}
    \end{figure*}

    \begin{figure*}
    \centering
	\begin{subfigure}[b]{0.3\textwidth}
	    \includegraphics[width=\textwidth]{"../Miniprojet 2.0/Part B/B_3_bifurcation.png}
	    \caption{$\lambda_1=1$ $\lambda_2=1.15$ \\
	    and initial conditions $X_{i,0}=0~Y_{i,0}=0~Z_{i,0}=3~V_0={i,0}$}
	\end{subfigure}
	~
	\begin{subfigure}[b]{0.3\textwidth}
	    \includegraphics[width=\textwidth]{"../Miniprojet 2.0/Part B/B_3_bifurcation_2.png}
	    \caption{$\lambda_1=1$ $\lambda_2=1.5$ \\
	    and initial conditions $X_{i,0}=0~Y_{i,0}=0~Z_{i,0}=3~V_0={i,0}$}
	\end{subfigure}
	~ 
	\begin{subfigure}[b]{0.3\textwidth}
	    \includegraphics[width=\textwidth]{"../Miniprojet 2.0/Part B/B_2_marginally_less_stupid.png}
	    \caption{$\lambda_1=1$ $\lambda_2=1.15$ \\
	    and initial conditions $X_{i,0}=0~Y_{i,0}=0~Z_{i,0}=0~V_{i,0}=0$}
	\end{subfigure}

	\caption{Bifurcation diagram in a two-cells Model $X_{min}$ and $X_{max}$ plotted at time intervals $[9/10; 1]$ of $t_{max}=2000$h. We observe that if $K>$ a threshold value and if the difference in the periods of the two cells is high enough, the circadian behaviour of both cells dies. The reason for this is that the positive feedback loop has high sensitivity and tends to overload concentrations of $Z_i$ which in turn inhibit any production of $X_i$. This behaviour is further noticeable when the initial conditions are far from those of their limit cycles, as the cells enter tighter limit cycles ($\Delta X_{(c)} < \Delta X_{(a)}$)}
	\end{figure*}


    \begin{figure*}
    \centering
	\begin{subfigure}[b]{0.45\textwidth}
	    \includegraphics[width=\textwidth]{"../Miniprojet 2.0/Part B/100cells/100cells0_0.png}
	    \caption{$K=0.0$}
	\end{subfigure}
	~ 
	\begin{subfigure}[b]{0.45\textwidth}
	    \includegraphics[width=\textwidth]{"../Miniprojet 2.0/Part B/100cells/100cells0_5.png}
	    \caption{$K=0.5$}
	\end{subfigure}
	~ 
	\begin{subfigure}[b]{0.45\textwidth}
	    \includegraphics[width=\textwidth]{"../Miniprojet 2.0/Part B/100cells/100cells1_0.png}
	    \caption{$K=1.0$}
	\end{subfigure}
	 
	\begin{subfigure}[b]{0.45\textwidth}
	    \includegraphics[width=\textwidth]{"../Miniprojet 2.0/Part B/100cells/100cells1_5a.png}
	    \caption{$K=1.5$}
	\end{subfigure}
	~ 
	\begin{subfigure}[b]{0.45\textwidth}
	    \includegraphics[width=\textwidth]{"../Miniprojet 2.0/Part B/100cells/100cells1_5b.png}
	    \caption{$K=1.5$}
	\end{subfigure}

	\caption{$X_i(t)$ trajectories in a 100-Cells Model with initial conditions $X_{i,0}=0~Y_{i,0}=0~Z_{i,0}=3~V_{i,0}=0$ and $\lambda_i \sim N(1, 0.05)$. As expected, with no coupling the cells are unable to synchronize leading to the loss of the signal in (a). Furthermore, with increasing $K$ the population synchronizes in longer periods. But the higher the strength of the intercellular signal, the more synchronisation we observe.}
    \end{figure*}


    \begin{figure*}[!htb] 		%oh my god this is so ugly pls don't look at me
	\captionsetup{labelformat=empty}
	\caption{\Huge{\textbf{Part C - Circadian Behaviour in the Brain}}}
    \end{figure*}

    \begin{figure*}
    \centering
	\begin{subfigure}[b]{\textwidth}
	    \includegraphics[width=\textwidth]{"../Miniprojet 2.0/Part C/C_2_figure_1.png}
	    \caption{Association matrix of all zeros / No cells are synchronised with each other.}
	\end{subfigure}
	 
	\begin{subfigure}[b]{\textwidth}
	    \includegraphics[width=\textwidth]{"../Miniprojet 2.0/Part C/C_2_figure_2.png}
	    \caption{Association matrix of all ones / All cells are synchronised.
	\end{subfigure}
	\caption{The above figures are the result of a 240h simulation. the heatmaps on the left show a geometrical representation of the SCN, with the colors indicating the concentration of X at the last timepoint.}
    \end{figure*}

\end{document}
