	\section{Introduction}
    The periodic changes that come about due to the day-night cycle to things ranging from visibility to temperature create an environment in which it is advantageous to adapt behaviour to this extrinsic period. Perhaps vision becomes so restricted in the night that basic processes like hunting for food or looking for mates become so inefficient that it's a better strategy to conserve energy when it's dark. Perhaps the opposite is the case and hunting during the night becomes an advantage due to specially evolved senses in animals such as, for example, bats. \\
    
    For this reason many different lifeforms, from bacteria to animals, have adapted to the 24 hour period that exists around them by synchronising behaviour and processes in different tissues to the day-night cycle by keeping track of time using internal 'clocks'. \\
    
    In humans this clock works by having a part of the brain, called the suprachiasmatic nuclei (SCN), manage the production of several clock proteins in such a way as to create oscillatory patterns in the concentration of these proteins. These patterns are called circadian when they follow the 24-hour period of the earth's rotation. \\
    
    Negative feedback loops are used in such a way as to create a periodic rise and fall in the production of proteins that can be influenced by external factors such a light intensity and food uptake to regularly adjust the synchronisation to new environments. \par

	\section{The Model}
In this simplified model of the SCN we do not take into account such outside signals and instead focus on behaviours of cells in the SCN all by themselves. The model looks at the production of clock gene mRNA (designated as X), which leads to production of a clock protein(Y). This clock gene protein activates transcription of a third compound (Z), which acts as a transcriptional inhibitor of X, thus creating a negative feedback loop. \\
    
    In this report we examine several conditions that need to be met for this simple model to exhibit circadian oscillations in the concentrations of the compounds involved. This model is then expanded to investigate synchronisation between cells by taking into account a third protein (V) that is exported and allows the cells in the SCN to talk to each other. First taking into account only 2 cells, this model is expanded to hundreds of cells later on, not reaching the human population of cells in the SCN of 10'000, but at least providing a smaller model that could be scaled up to that number if more processing power was available. \par
    
    \section{Analysis of the model A}
    The model under investigation in this report will start out very simple, with a large number of shortcomings and assumptions, which we will attempt to improve upon as analysis progresses. At first we consider a single cell, in a very simple attempt to find conditions under which the concentrations of X,Y and Z will display circadian oscillation. In this first part we will especially focus on the central parameter of the base translation rate of X, the clock gene mRNA. We will vary its value and observe under which conditions circadian behaviour appears. This model obviously looks at one cell in isolation and fails to capture the main usefulness of the system, its ability to integrate information from outside. \par
    
    \section{Analysis of the model B}
    In section B we will include multiple cells and the production of the protein V, which allows cells to communicate and synchronise. In this section we will assume that V diffuses so quickly that we can assume its concentration to be the same around all the cells we look at, which is obviously a significant simplification. We would also assume all kinds of delays in its mode of action, depending on how its concentration is sensed by the cells. Of special interest in this section are intrinsic periods of cells. It is sensibly assumed that not all cells will have perfectly identical production rates, and that ultimately the sum of all these differences will be apparent in the period of their oscillations. Next to that we will investigate the coupling factor K, which allows us to manipulate how strongly the cells respond to the average extracellular concentration of V, denoted as F. The strength of the coupling factor, as well as the difference in intrinsic periods of the cells have an impact on whether the cells will synchronise or not, and this section attempts to shed light on these respective conditions. 
    
    
    